\documentclass[12pt]{article}
\usepackage{indentfirst}

\title{MLRTEMS: Embedded Standard-ML}
\author{Nathan Burgers}

\begin{document}
\maketitle

\section{Summary}
Functional programming is often ignored in embedded and real-time systems because of its perceived runtime overhead. Our project aims to provide a bridge between the MLton optimizing Standard-ML compiler, and the RTEMS real-time systems executive.

\section{Project Overview}
Standard-ML is a well-documented and standardized functional programming language with a strong, inferred Hindley-Milner type system. We argue that embedded systems could stand to benefit immensely from functional constructs such as immutable data structures and pattern matching provided in languages like Standard-ML. Embedded systems also require strict run-time performance guarantees, and Standard-ML provides mutable references and arrays to accommodate times when one would want to sacrifice purity for speed.\\
\indent MLton translates Standard-ML code to optimized C and RTEMS is provided as a series of C libraries that are conditionally compiled to meet the needs of the system. In order to get a basic embedded system running Standard-ML, a set of primitives bridging RTEMS's threading system are needed

\end{document}